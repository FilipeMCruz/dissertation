
\chapter{Sensae Console Database Configuration}
\label{appendix:implementation:description:database}

The solution designed relies on various databases, and as discussed in Section~\ref{subsubsec:implementation:decisions:database:relational} some are relational databases. \citetitle{postgressql} and most databases of this data-model type require a database schema. For this solution the schema of each database is defined in a \textit{sql} file that is executed at the start of the database, only if no data is found.

Further database schema migrations are preformed using custom \gls{SQL} scripts when needed. In the future, once more instance of \textbf{Sensae Console} are deployed, the use of liquidbase or flyway is preferred.

The following Code Sample~\ref{code:implementation:description:database:file} exemplifies the content of this scripts.

\begin{lstlisting}[language=SQL, caption=Initialization Script Segment for Data Processor Database, label={code:implementation:description:database:file}]
create table if not exists public.transformation
(
    persistence_id bigint generated by default as identity
        primary key,
    device_type    varchar(255)
        constraint unique_type_constrain
            unique
);

create table if not exists public.property_transformation
(
    persistence_id                bigint generated by default as identity (maxvalue 2147483647)
        primary key,
    value                         integer           not null,
    old_path                      varchar(255),
    transformation_persistence_id bigint
        constraint ref_transformation_constrain
            references public.transformation,
    sub_sensor_id                 integer default 0 not null
);
\end{lstlisting}

This script defines two simple tables, \textit{transformation} and \textit{property\_transformation}, following the concepts defined in Section~\ref{subsubsec:design:domain:bounded_contexts:processor}.

Apart from the schema, the \textbf{Identity Management Database} also requires the following bootstrap data, as implied in \nameref{subsubsec:design:domain:bounded_contexts:identity} Bounded Context Section:

\begin{itemize}
    \item Root domain;
    \item Public domain;
    \item Unallocated Root domain;
    \item Anonymous Tenant account;
    \item Admin Tenant account;
\end{itemize}

This data is inserted using the following function, Code Sample~\ref{code:implementation:description:database:function}:

\begin{lstlisting}[language=SQL, caption=Bootstrap function for Identity Management Database, label={code:implementation:description:database:function}]
CREATE FUNCTION public.init_domains ()
RETURNS varchar(255) AS $root_oid$
    DECLARE
        root_oid varchar(255) := gen_random_uuid();
        public_oid varchar(255) := gen_random_uuid();
        unallocated_oid varchar(255) := gen_random_uuid();
    BEGIN
        INSERT INTO public.domain (name, oid, path)
        VALUES ('root', root_oid, ARRAY[root_oid]);
        INSERT INTO public.domain (name, oid, path)
        VALUES ('public', public_oid, ARRAY[root_oid, public_oid]);
        INSERT INTO public.domain (name, oid, path)
        VALUES ('unallocated', unallocated_oid, ARRAY[root_oid, unallocated_oid]);
        INSERT INTO public.tenant (name, oid, phone_number, email, domains)
        VALUES ('Anonymous', gen_random_uuid(), '', '', ARRAY[public_oid]);
        INSERT INTO public.tenant (name, oid, phone_number, email, domains)
        VALUES ('Admin', gen_random_uuid(), '', '$SENSAE_ADMIN_EMAIL', ARRAY[root_oid]);
        RETURN root_oid;
    END;
$root_oid$ LANGUAGE plpgsql;

select public.init_domains();

DROP FUNCTION public.init_domains;
\end{lstlisting}

This function starts by declaring three \gls{UUID} - lines \textbf{4} to \textbf{6} - that will later be used to populate the domain's \textit{path} and the tenant's \textit{domains} - lines \textbf{7} to \textbf{17}. In the end the function is executed and then removed to ensure that it isn't executed again.

In line \textbf{17}, the variable \textbf{\$SENSAE\_ADMIN\_EMAIL} is replace by a valid email before building the database container with the full script. This variable configuration is discussed in the Section~\ref{subsec:implementation:description:config}.
