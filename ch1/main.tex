\chapter{Introduction}
\label{chap:introduction}

This chapter provides a short introduction to this dissertation. It describes this work's context, the problem it addresses, the objectives to be fulfilled, the approach taken and finally the results achieved. The chapter closes with the document's structure.

\section{Context}
\label{sec:introduction:context}

The \gls{IoT} is a fast-growing technological concept, which aims to integrate various physical and virtual objects into a global network to enable interaction and communication between those objects \parencite{Atzori2010TheIO}. According to \cite{NIZETIC2020122877} the main goal of \gls{IoT} technologies is to simplify processes in different fields, to ensure a better efficiency of systems (technologies or specific processes) and finally to improve life quality.
Currently many large-scale enterprises use custom-made \gls{IoT} technologies to aid their decision making. For example:

\begin{itemize}
    \item Ericsson has created a platform, Ericsson Maritime ICT, designed to collect and present data regarding cargo ships. Sensors capture information regarding the speed and location of the ship as well as the temperature and condition of the reefer containers. This information is updated in real time and presented to the various parties in the supply chain \parencite{ericson-marinetime};
    \item John Deere has created the JDLink platform, designed to give farmers live information about their fleet's location as well as diagnostic and usage data for each machine. Sensors that measure soil and crop conditions in real time help farmers to decide the best time to start harvesting \parencite{jdlink};
    \item Verizon has created a platform, Verizon Connect, designed to help reduce fuel consumption, monitor vehicle diagnostics \& vehicle maintenance needs, prevent unauthorized out of area use and much more. Sensors installed by Verizon in cars, trucks and machines give insights in real-time about the fleet \parencite{verizon-iot}.
\end{itemize}

Like these, many other large companies are building platforms to aid decision making based on sensor data harvesting.
In a pursuit for sustainability, companies are looking to \gls{IoT} as an approach to increase efficiency and decrease waste. According to \cite{BIBRI2018230} the \gls{IoT} and related big data applications can play a key role in catalyzing and improving the process of environmentally sustainable development.

Some of the benefits that \gls{IoT}, and these platforms, bring to companies are: more operational efficiency, increased security conditions, and cost reduction \parencite{forbes-why-iot}.

\section{Problem}
\label{sec:introduction:problem}

Despite the promised benefits, the initial investment this technology requires to be employed is very high for small and medium companies. As such, its adoption is often postponed or discarded.

In addition to the high costs, these platforms are often associated with a company and its products or businesses, for example, according to \cite{6851114} in China most \gls{IoT} applications are domain-specific or application-specific solutions. Another study by \cite{noura2019interoperability} determine that vendor lock-in is a real concern in \gls{IoT}, quoting: ``each solution provides its own IoT infrastructure, devices, APIs, and data formats leading to interoperability issues''.
This is often a problem. As an example, for small farmers it is economically unthinkable to change machines and fleet just to be able to benefit from these services.

A service that acts upon \gls{IoT} data is composed of many pieces and processes, such as (i) managing device network connectivity and ownership, (ii) capturing data via sensors, (iii) routing data through the network, (iv) aggregating and storing data, (v) transforming data into concise information, (vi)  analyzing the information captured, (vii) triggering alarms based on this analysis, (viii) providing the gathered information visually or programmatically. It's a complex and constantly evolving system.

In order to deal with these needs there are platforms on the market that facilitate the creation of these services by taking care of device connection and management, such as \citetitle{aws-iot}, \citetitle{azure-iot}, \citetitle{google-iot} and others. Their main purpose is to act as a middleware between costumer-facing application and physical \textit{things} deployed somewhere, such as sensors, actuators or hybrid devices. Each service provides a set of additional functionalities such as data visualization, transformation, storage and analysis.

However these platforms don't provide pre-made specialized solutions to aid the decision making process of end customers and small businesses, such as fleet management, smart irrigation, tracking of deliveries and goods, indoor fire detection, and others. This is often a problem to companies that have little to no background in IoT and in software development. As an example, for a small transportation company it's unthinkable to resort to this middleware services in order to create a fleet management system and perceive the benefits \gls{IoT} can provide.

Due to this obstacles the adoption of \gls{IoT} technologies by small companies and individuals is lingered. According to \cite{iot-fail}, 60\% of \gls{IoT} projects stall at the \gls{PoC} stage.

\section{Objectives}
\label{sec:introduction:objectives}

This work idealizes the creation of a platform responsible for further facilitating the creation of \gls{IoT} based services. It must focus on:

\begin{itemize}
    \item Agnostically interacting with different \gls{IoT} middlewares (receiving sensor measures and dispatching commands to actuators through these platforms);
    \item Homogenizing and sanitizing the device information, commands available and measures received in a single concise form and semantic;
    \item Providing various means to interact with the platform and the information handled by it, depending on the costumer needs, such as: (i) full-fledged access via \gls{UI}, (ii) high-level \gls{API} focused on its core functionalities, (iii) low-level and generic \gls{API} to consume device measures and alerts.
\end{itemize}

To answer these high-level objectives, the platform should encompass essential functionalities such as:

\begin{itemize}
    \item Data Aggregation: responsible for providing a simple entry-point to the system for any \gls{IoT} middleware;
    \item Data Filtering: responsible for discarding erroneous device measures;
    \item Data Retention: responsible for storing the device measures received;
    \item Data Transformation: responsible for processing unsanitized data and extracting relevant information from it;
    \item Data Presentation: responsible for swiftly presenting information to the user;
    \item Trigger Warning System: responsible for dispatching alerts based on rules applied to the data in motion;
    \item User Authentication/Authorization: responsible for allowing/denying access to the various platform's components and data depending on the user authentication and authorization level.
\end{itemize}

Finally, this project envisages the creation of \gls{PoC}s that answer specific business cases and utilize the developed platform. These \gls{PoC}s can follow distinct approaches for user interactions: from a full-fledged \gls{UI}, a simple and business case focused \gls{API}, or a basic service that dispatches emails/SMS based on alerts captured.

Some of the business cases to address, and their main requirements, are:

\begin{itemize}
    \item Fleet Management: fleet location feed, fleet location history, calculation of distance traveled by the fleet;
    \item Smart Irrigation: storage and presentation of environmental conditions captured by sensors and automatic activation of the irrigation system via commands sent to actuators;
    \item Indoor Fire Outbreak Surveillance: room conditions, alarm trigger system based on abnormal conditions;
    \item Smart Parking: ongoing information regarding free and occupied parking slots.
\end{itemize}

As such, this project's tangible objectives can be tracked and measured according to two conceptual axis.
An axis is related to the platform and its core functionalities (that any service, specific to a business case, relies on) and requirements (being agnostic to \gls{IoT} middlewares, defining a semantically sound and homogeneous data model, offering different user-faced means of interaction). The other axis is related to the \gls{PoC}s focused on specific business cases.

\section{Approach}
\label{sec:introduction:approach}

This work is a greenfield project with the intent of designing and implementing a platform that simplifies the creation of applications based on \gls{IoT} captured and analysed data and interactions with actuators. Some \gls{PoC}s that answer the needs of various business cases must be developed.
Each business case is considered a concern and should be addressed in an independent \gls{PoC}.
In the end of the project, the envisaged platform will support the intended \gls{PoC}s.

The pursued approach envisions the project divided in four phases:

\begin{itemize}
    \item Phase I: Design and implement \gls{PoC}s that support each business case;
    \item Phase II: Identify commonality and variability between all designed prototypes;
    \item Phase III: Design and implement a platform that simplifies the development of this \gls{PoC}s by aggregating common needs and concepts;
    \item Phase IV: Refactor the \gls{PoC}s so that they rely on the platform's functionalities.
\end{itemize}

During the first phase it is extremely important that the design and implementation of each \gls{PoC} takes into account the goals of phase II. Even though these services are independent they all shared core responsibilities, functionalities and procedures that can be reused.

During the second phase, the various \gls{PoC}s will be evaluated so that most common components can be moved to the platform and later reused by them.

During the third phase, a platform that comprises shared functionalities of all \gls{PoC}s' business cases must be designed and implemented. This platform must offer an agnostic, homogeneous and concise access to sensors and actuators regardless of the \gls{IoT} platform used to connect to them.

In the final phase, the developed \gls{PoC}s must be integrated with the \gls{API} provided by the platform.

The project management will adopt the Scrum methodology described by \cite{schwaber1997scrum}, with monthly sprints that end in presentations of the software to the company and weekly meetings focusing on reviewing the progress, discussing issues that rose and future ideas to add to the backlog.

\section{Achieved Results}
\label{sec:introduction:achieved_results}

This work gave birth to a platform, \textbf{Sensae Console}, capable of handling the desired requirements and functionalities. Some of its main features are: (i) powerful data classification and categorization, (ii) custom data manipulation via scripting, (iii) virtual device registry and ownership, (iv) integrated rule engine to dispatch alerts, (v) strict user authentication and authorization, (vi) rich set of GraphQL \gls{API} for management, (vii) designed to scale, (viii) designed to incorporate third-party services as plugins, (ix) flexible hosting options: multi-tenant or dedicated.

This platform proved itself capable of integrating with the researched \gls{IoT} middlewares while offering to consumers a semantically rich and homogeneous data model. The concepts tackled by this data model were materialized in a \gls{SDK}, \textit{iot-core}, that was created to ease the development of services integrated with this platform via the low-level, generic and event-based \gls{API}.

Three \gls{PoC}s were designed and implemented during this project time span: (i) fleet management, (ii) smart irrigation and (iii) notification management.

The platform and \gls{PoC}s were later evaluated according to the performance requirements envisioned for them as a dedicated hosted solution.

\section{Document Structure}
\label{sec:introduction:document_structure}

This document is divided into 7 more chapters that explore this work.

\begin{itemize}
    \item \nameref{chap:stateofart}: where literature related to this work is explored;
    \item \nameref{chap:requirements}: where this project's requirements are listed;
    \item \nameref{chap:design}: where the architectural design of the solution is presented;
    \item \nameref{chap:implementation}: where the implementation of the solution is addressed;
    \item \nameref{chap:evaluation}: where the evaluation of the solution is presented and results discussed;
    \item \nameref{chap:conclusion}: where a final overview of the project is presented, wrapping up the achievements and future work of this project and solution.
\end{itemize}
