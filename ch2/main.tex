\chapter{State of the Art}
\label{chap:stateofart}

This chapter introduces a modest introduction to the \gls{IoT} landscape. Focusing first on technologies, solutions and the architectural context surronding them and later the various business areas where \gls{IoT} is required.
The intent of this chapter is to introduce the reader to the subjects related to this work.

\section{Internet of Things}
\label{sec:stateofart:iot}

%todo
This section has provided a short overview of IoT devices and systems. It is not designed to be comprehensive or even extensive, but simply to provide enough background to support the discussion of requirements and capabilities below.

\section{IoT Architectural Context}
\label{sec:stateofart:arch}

This section focus on the landscape of services, solutions, tools, terms and technologies that are related to \gls{IoT}.
It starts by dividing the concepts in three distinct categories:

\begin{itemize}
    \item \nameref{subsec:stateofart:arch:infra}: concepts that don't answer any specific business case but are common or even a requirement for any \gls{IoT} related project; 
    \item \nameref{subsec:stateofart:arch:platforms}: concepts that support and ease the creation of \gls{IoT} projects but are not a necessity;
    \item \nameref{subsec:stateofart:arch:solutions}: concepts that answer specific business cases.
\end{itemize}

After these categories are explored, some \nameref{subsec:stateofart:arch:ref} are presented and correlated with some of the concepts mentioned before.

\subsection{Infrastructure} % Integration?
\label{subsec:stateofart:arch:infra}

\subsubsection{Mediator}
\label{subsubsec:stateofart:arch:infra:mediator}

\subsubsection{IoT Midlleware}
\label{subsubsec:stateofart:arch:infra:middleware}

\subsubsection{Asynchronous Communication}
\label{subsubsec:stateofart:arch:infra:async}

\subsubsection{Data Storage}
\label{subsubsec:stateofart:arch:infra:store}

\subsubsection{Rule Engine}
\label{subsubsec:stateofart:arch:infra:rule}

\subsection{Platforms}
\label{subsec:stateofart:arch:platforms}

\subsection{Solutions}
\label{subsec:stateofart:arch:solutions}

\subsection{Reference Architectures}
\label{subsec:stateofart:arch:ref}

As the \gls{IoT} domain covers such a wide spectrum of application fields with very little in common, the development cycles, technologies and architectures used can be completely different. This section sheds a light on some of the Reference Architectures of this field, what their focus is and the relevancy they have for this project.

The Reference Architectures discussed are: (i) \nameref{subsubsec:stateofart:arch:iota}, (ii) \nameref{subsubsec:stateofart:arch:sat}, (iii) \nameref{subsubsec:stateofart:arch:iira}, (iv) \nameref{subsubsec:stateofart:arch:wso2}, (v) \nameref{subsubsec:stateofart:arch:p2413}, (vi) \nameref{subsubsec:stateofart:arch:rami}, (vii) \nameref{subsubsec:stateofart:arch:intel}, (viii) \nameref{subsubsec:stateofart:arch:azure}.

This section was based on the papers written by \cite{Lynn2020} and \cite{DIAS2022100529}.

\subsubsection{IoT-A}
\label{subsubsec:stateofart:arch:iota}

The IoT Architecture goals are to create ``an architectural reference model for the interoperability of Internet-of-Things systems, outlining principles and guidelines for the technical design of its protocols, interfaces, and algorithms'' that shall ``lead to corresponding mechanism for its efficient integration into the service layer of the Future Internet''. \parencite{iot-a}. This project's final version is dated back to July 2013.

It defines a collection of Unified Requirements that support and validate concrete architectures created according to the \gls{IoT} \gls{ARM}. Some of these requirements were also applied to this project.

According to \cite{krvco2014designing}, this project motto is ``to offer IoT architects a common technical grounding in order to optimize interoperability. In that case, IoT applications would not be any longer built as stand-alone silo applications, but as inter-operable vertical applications still having a common "horizontal" grounding - the \gls{ARM} (compliant components, protocol suites, etc.)''.

This project originated the \gls{IoT} \gls{ARM} that can be divided into three interconnected parts \parencite{krvco2014designing}:

\begin{itemize}
    \item The IoT Reference Model;
    \item The IoT Reference Architecture;
    \item A set of Guidance (also called best practice).
\end{itemize}

The reference model defines several models that help to describe certain aspects of the \gls{IoT} architecture, some of this models are described by \cite{6682101}. It defines five core concepts:

\begin{itemize}
    \item Augmented Entity: a combination of a Physical Entity (real world object) and its Virtual Entity (digital representation);
    \item User: Those who interact with the system, human beings, devices, services and others;
    \item Device: Hardware to monitor or interact with real world objects;
    \item Resource: Computational element that gives access to information or control over a real-world object;
    \item Service: Entities that expose resources via a common interface, making then available for consumption by other services;
\end{itemize}

Each concept is them explored in detail. These concepts are them used to create the reference architecture.

According to \citetitleyear{iot-arm}, ``the Reference Architecture can be visualized as the "Matrix" that eventually gives birth ideally to all concrete architectures''. ``Guidance in form of best practices can be associated to a reference architecture in order to derive use-case-specific architectures from the reference architecture''.

The IoT Reference Architecture provides a functional view, presented in the Figure~\ref{fig:stateofart:arch:iota:functional}.

\begin{figure}[H]
    \centering
    \includegraphics[scale=0.5]{
        assets/figures/arm-functional-view.png
    }
    \caption[ARM Functional View]{ARM Functional View, \cite{iot-arm-functional}}
    \label{fig:stateofart:arch:iota:functional}
\end{figure}

This functional view divides the architecture of a system in various functional components with well defined concerns.

\subsubsection{SAT-IoT}
\label{subsubsec:stateofart:arch:sat}

\cite{8767282} present an architectural model definition that lead to the development of ``a new advanced \gls{IoT} platform referred as SAT-IoT''. This model attempts to integrate concepts such as:  ``the paradigm of edge/cloud computing transparency, the IoT computing topology management, and the automation and integration of IoT visualization systems''. This project's final version is dated back to 2019, it apeares that the envisioned platform was not implemented since no other reference to it was found.

The diagram in Figure~\ref{fig:stateofart:arch:sat:model} defines the concepts, services and relations of this model.

\begin{figure}[H]
    \centering
    \includegraphics[scale=0.4]{
        assets/figures/sat-iot.png
    }
    \caption[SAT-IoT Architectural Model]{SAT-IoT Architectural Model, \cite{8767282}}
    \label{fig:stateofart:arch:sat:model}
\end{figure}

This model focus on a distributed system with a tree-like structure, where data can be dynamically processed in different node levels (edge, mid or cloud) in order to optimize response latency, bandwidth consumption, storage and other metrics. Components such as "IoT Topology Management Entity", "IoT Data Flow Dynamic Routing Entity" and "IoT Visualization Entity" focus on optimally distributing the workload across all nodes of the system.

The architecture also enables one to host "Embedded IoT Applications" that have full access to the system internals, leading to strongly integrated applications.

\subsubsection{IIRA}
\label{subsubsec:stateofart:arch:iira}

The \gls{IIRA} ``addresses the need for a common architecture framework to develop interoperable IIoT systems for diverse applications across a broad spectrum of industrial verticals in the public and private sectors to achieve the true promise of IIoT'' \parencite{iira}. This project's final version is dated back to June 2019.

It decomposes a typical Industrial \gls{IoT} system in five distinct functional domains:

\begin{itemize}
    \item \textbf{Control Domain}: this domain focus on functions that are performed by industrial control and automation systems. It is deployed in proximity to the physical systems and therefore geographically distributed;
    \item \textbf{Operations Domain}: this domain focus on the management and operation of the control domain. It should be able to configure, register, track and control assets. It is also responsible for providing real-time prognostics, monitoring and diagnostics of the managed assets;
    \item \textbf{Information Domain}: this domain is responsible for managing and processing data, it should transform, persist, and model or analyze data to acquire high-level intelligence about the overall system;
    \item \textbf{Application Domain}: this domain is responsible for applying business focused rules and logic to the gathered information;
    \item \textbf{Business Domain}: this domain is responsible for implementing business processes, such as Enterprise Resource Planning, Costumer Relationship Management, Manufacturing Execution System, Billing and Payment, Work Planning and Scheduling Systems. 
\end{itemize}

These domains interact according to Figure~\ref{fig:stateofart:arch:iira:domains}.

\begin{figure}[H]
    \centering
    \includegraphics[scale=0.5]{
        assets/figures/irra-domains.png
    }
    \caption[\gls{IIRA} Functional Domains]{\gls{IIRA} Functional Domains, \cite{iira}}
    \label{fig:stateofart:arch:iira:domains}
\end{figure}

As information flows from the control domain to the business domain it is enriched, cleaned, filtered and combined leading to a broader and richer notion of the complete environment. New information can be derived, and new intelligence may emerge from this broader information.

When applied to the common three tier architecture for IoT systems, these domains are organized according to Figure~\ref{fig:stateofart:arch:iira:applied}.

\begin{figure}[H]
    \centering
    \includegraphics[scale=0.5]{
        assets/figures/irra-applied.png
    }
    \caption[Mapping between a three tier architecture and the \gls{IIRA} function domains]{Mapping between a three tier architecture and the \gls{IIRA} function domains, \cite{iira}}
    \label{fig:stateofart:arch:iira:applied}
\end{figure}

\subsubsection{WSO2 IRA}
\label{subsubsec:stateofart:arch:wso2}

The WSO2 reference architecture aims to ``provide an architecture that supports integration between systems and devices'' \parencite{wso2ira}. This project's final version is dated back to October 2015.

It groups the \gls{IoT} related requirements in the following key categories: (i) Connectivity and communications, (ii) Device management, (iii) Data collection, analysis, and actuation, (iv) Scalability, (v) Security, (vi) high-availability, (vii) Predictive analysis and (viii) Integration.

This categories gave birth to the following reference architecture, Figure~\ref{fig:stateofart:arch:wso2:ra}.

\begin{figure}[H]
    \centering
    \includegraphics[scale=0.4]{
        assets/figures/wso2-ira.png
    }
    \caption[WSO2 Reference Architecture for IoT]{Reference Architecture for IoT, \cite{wso2ira}}
    \label{fig:stateofart:arch:wso2:ra}
\end{figure}

This reference architecture envisions two cross-cutting and five horizontal layers:

\begin{itemize}
    \item \textbf{Device Layer} (in grey): related to the physical \gls{IoT} devices;
    \item \textbf{Communications Layer} (without any representative color): related to the connectivity of devices;
    \item \textbf{Aggregation/bus Layer} (in light blue): related to the aggregation and supply of data upstream, bridging between the protocols used in upstream layers and downstream layers; 
    \item \textbf{Event processing and Analytics Layer} (in purple): related to data processing, analysis and storage;
    \item \textbf{Client/external Communications Layer} (in green): related to web-based frontends and portals that interact with the Event processing and Analytics layer, dashboards that
    offer views into analytics and event processing, and \gls{API}s for machine to machine communication;
    \item \textbf{Device Management Layer} (in dark blue): related to the management, onboarding and remote control of devices;
    \item \textbf{Identity Access Management} (in orange): related to the authentication and authorization of users and systems that interact with the system.
\end{itemize}

In the Event processing and analytics Layer it's recommended the use of ``a highly scalable, column-based data storage for storing events'', ``map-reduce for long-running batch-oriented processing of data'', ``complex event processing for fast in-memory processing and real-time reaction and autonomic actions based on the data an activity of devices and other systems'' and ``traditional application processing platforms'' (custom-made applications for data processing).

\subsubsection{P2413}
\label{subsubsec:stateofart:arch:p2413}

The IEEE Standard for an Architectural Framework for the IoT ``defines an architecture framework description for \gls{IoT}''. The architecture framework defined in the standard ``will promote cross-domain interaction, aid system
interoperability and functional compatibility, and further fuel the growth of the IoT market'' \parencite{9032420}.  This project's final version is dated back to June 2020.

Its architecture framework covers the definition of basic architectural building blocks and their ability to be integrated into multi-tiered systems. It describes different detailed viewpoints of the framework \parencite{9032420}:

\begin{itemize}
    \item \textbf{Conceptual Viewpoint}: concerned with defining a common vocabulary and semantics regarding a \gls{IoT} System to ease the communication across teams and encourage the reuse of concepts;
    \item \textbf{Compatibility Viewpoint}: concerned with the compatibility between systems and devices to lower the cost of integration. This viewpoint urges for the creation of new standards and compliance with those standards. It defines six levels of compatibility focused specially on physical devices: (i) incompatible, (ii) coexistent, (iii) inter connectable, (iv) inter workable, (v) interoperable and (vi) exchangeable;
    \item \textbf{Lifecycle Viewpoint}: concerned with a system's assurance, performance, maintainability and evolvability across its lifecycle: design, development, production, support, upgrade and retirement; 
    \item \textbf{Communication Viewpoint}: concerned with how devices can exchange information with each other and \gls{IoT} systems in a accurate, precise and effective manner;
    \item \textbf{Information Viewpoint}: concerned with how information is semantically defined, structured, stored, shared, manipulated, managed, and distributed across the IoT system. This viewpoint should focus on documenting system-level information, e.g., information exchanged between the various subsystems;
    \item \textbf{Function Viewpoint}: concerned with how devices can function according to their intended purpose or characteristic action, such as actuation, sensing, analysis, or control of entities of interest; 
    \item \textbf{Thread model Viewpoint}: concerned with identifying potential threats that could exploit vulnerabilities in the device, network or subsystems that encapsulate the \gls{IoT} system;
    \item \textbf{Security and safety monitoring Viewpoint}: concerned with monitoring the events occurring in an \gls{IoT} system and analyzing them for signs of possible incidents, which are violations or imminent threats of violation of security, safety, or acceptable use policies, or standard security practices;
    \item \textbf{Access control Viewpoint}: concerned with permitted activities of legitimate users, mediating every attempt by a user to access a resource in the IoT system. It is composed by three security functions: identification, authentication and authorization;
    \item \textbf{Privacy and trust Viewpoint}: concerned with the privacy of individuals or groups and trust in systems or organizations. In a complex \gls{IoT} system, arbitrary device data can be grouped and analyzed to determine the users activities;
    \item \textbf{Collaboration Viewpoint}: concerned with the collaboration of systems that belong to different application domains;
    \item \textbf{Computing resource Viewpoint}: concerned with the computing resources needed to support the \gls{IoT} system as a whole, such as gateways, data centers, \gls{PLC}s and \gls{DCS} controllers, microcontrollers embedded in sensors and actuators or others.
\end{itemize}

The idealized \gls{IoT} System should be examined according to these viewpoints in order to better define its architecture.

\cite{9032420} then procedes to define the Standard for a Reference Architecture for Smart Cities in P2413.1, the major focus of this project's business cases.

One of the architectures proposed in the standard and derived from the architecture framework is presented in Figure~\ref{fig:stateofart:arch:p2413:rasc}. 

\begin{figure}[H]
    \centering
    \includegraphics[scale=0.4]{
        assets/figures/smart-city-p2413.png
    }
    \caption[Example of an \gls{IoT} System Architecture for Smart Cities]{Example of an \gls{IoT} System Architecture for Smart Cities, \cite{9032420}}
    \label{fig:stateofart:arch:p2413:rasc}
\end{figure}

This architecture, derived from the common three tier architecture for \gls{IoT} Systems, proposes a new tier entitled Smart City Platform, in it ``Northbound APIs support diverse vertical applications development and southbound APIs connect to different IoT Platforms'' \parencite{9032420}.

\subsubsection{RAMI 4.0}
\label{subsubsec:stateofart:arch:rami}

The Reference Architectural Model Industry 4.0 ``ensures that all participants involved share a common perspective and develop a common understanding'' and is represented by a ``three-dimensional map showing the most important aspects of Industrie 4.0'' \parencite{hankel2015reference}. It represents a service-oriented architecture according to the manufactures association that defined it. This project's final version is dated back to August 2018 and has clear focus on the \gls{IoT} business area related to the Industry, e.g. smart factories.

The three-dimensional map is depicted in Figure~\ref{fig:stateofart:arch:rami:map}.

\begin{figure}[H]
    \centering
    \includegraphics[scale=0.4]{
        assets/figures/rami.png
    }
    \caption[RAMI 4.0 Three-dimensional map]{RAMI 4.0 Three-dimensional map, \cite{rami}}
    \label{fig:stateofart:arch:rami:map}
\end{figure}

According to \cite{rami-explained} it is comprised of six architecture layers stretched across the hierarchy and life cycle axes:

\begin{itemize}
    \item \textbf{Business Layer}: concerned with Organization and Business processes;
    \item \textbf{Functional Layer}: concerned with the Functions of assets;
    \item \textbf{Information Layer}: concerned with the processing of the necessary data; 
    \item \textbf{Communication Layer}: concerned with how to gain access to the information needed;
    \item \textbf{Integration Layer}: concerned with the transition from things to the digital world;
    \item \textbf{Asset Layer}: concerned with the physical things in the real world.
\end{itemize}

This reference architecture mentions an administration shell that sits in between the asset (machine, sensor, unit or plant) and the network. This administration shell is the interface connecting the \gls{IoT} platform to the asset, storing all data and information about the asset and standardizing the network's communication. According to \cite{rami2}, ``each physical thing has its own administration shell'' and ``several assets can form a thematic unit with a common administration shell''. This administration shell allows for distributed data analysis and control over assets. 

According to \cite{iira-inter-rami}, this reference architecture is aligned with \nameref{subsubsec:stateofart:arch:iira}. The following picture, Figure~\ref{fig:stateofart:arch:rami:mapping} describes how RAMI 4.0 concepts can be represented according to IIRA.

\begin{figure}[H]
    \centering
    \includegraphics[scale=0.4]{
        assets/figures/iira-rami-mapping.png
    }
    \caption[IIRA and RAMI 4.0 Functional Mapping]{IRRA and RAMI 4.0 Functional Mapping, \cite{iira-inter-rami}}
    \label{fig:stateofart:arch:rami:mapping}
\end{figure}

\subsubsection{Intel SAS}
\label{subsubsec:stateofart:arch:intel}

\subsubsection{Azure IRA}
\label{subsubsec:stateofart:arch:azure}

\subsection{Synopsis}
\label{subsec:stateofart:arch:synopsis}

To close this section some of the gathered opinions and ideas surrounding these reference architecture models are presented. 

The author found the extendability notion behind the "Embedded IoT Application" component of \nameref{subsubsec:stateofart:arch:sat} interesting from a business point of view.

\section{Business Areas}
\label{sec:stateofart:areas}

Even though there's no concise structure, it is obvious that the \gls{IoT} technologies can be used in a broad range of areas/sectors. According to \cite{nivzetic2019smart}, the most valuable areas are: Smart Cities, Industrial \gls{IoT}, Connected Health and Smart Homes. The general market division of IoT technologies is presented in Figure~\ref{fig:iot-areas}.

\begin{figure}[H]
    \centering
    \includegraphics[scale=0.5]{
        assets/figures/iot-areas.png
    }
    \caption[IoT market structure]{General market structure of IoT technologies, \cite{nivzetic2019smart}}
    \label{fig:iot-areas}
\end{figure}

From another point of view, and according to \cite{7073822}, the sectors \gls{IoT} is related to are: Energy, Smart City, Transportation, Smart Home, Environment, Supply Chain, and Health Care.

According to (\cite{6851114}) these are the main application fields for \gls{IoT} in China: industry, smart agriculture, smart logistics, intelligent transportation, smart grids, smart environmental protection, smart safety, smart medical and smart home.

Even though this work focus mostly on Smart Cities other areas will also be described. Each of this areas incorporate several interconnected use cases that will briefly described in the following segments in accordance with \cite{nivzetic2019smart}.

\subsection{Smart Cities}
\label{subsec:stateofart:areas:cities}

The Smart Cities sector includes numerous use cases related to public safety, the environment, mobility, energy, infrastructure and many other municipal concerns. According to \cite{iot-smart-city-prioritized} this are the use cases being prioritized.

\begin{itemize}
    \item Connected Public Transport: real-time monitoring of public transportation vehicles' locations, stops and itineraries, and the possibility to be notified when a public transportation vehicle is arriving at a stop;
    \item Traffic Monitoring and Management: real-time monitoring and management of traffic flows in a efficient manner;
    \item Water level / Flood Monitoring: real-time monitoring of level of water in public water basins such as rivers, channels, or even lakes and seas to warn and predict fast water level shifts;
    \item Video Surveillance \& Analytics: real-time monitoring using \gls{CCT} cameras and analytics to detect specific situations, e.g. accidents, crimes, potential threats, or recognize specific features (face recognition, demographics, etc.);
    \item Connected Streetlights: real-time monitoring and management of streetlights' health status and energy consumption to decrease costs and become more sustainable;
    \item Weather Monitoring: real-time monitoring of weather conditions such as temperature, humidity, rainfall, wind speed and direction to predict the weather and future natural disasters;
    \item Air Quality / Pollution Monitoring: real-time monitoring of air quality to warn the community about hazardous conditions;
    \item Smart Metering - Water: remote real-time monitoring of water usage in homes to address the world's water demand and scarcity issues and faster localize sewage leaks;
    \item Fire / Smoke Detention: real-time monitoring of possible indoor fires and CO2 levels to prevent injuries, fatalities and building degradation;
    \item Water Quality Monitoring: real-time monitoring of water conditions such as pH levels, percentage of salts and other elements that can threaten the public health.
\end{itemize}

Apart from these use cases, others are arising, such as smart parking (\cite{GOAP201841}), smart irrigation (\cite{7562735}) and waste management (\cite{7972276}).
\begin{itemize}
    \item Smart parking provides a simple method to the community of knowing the available parking spots, which, alone, lowers the carbon footprint and traffic congestions in cities.

    \item Smart irrigation tackles the need to save water by irrigating the soil only when needed and not when it is already moist, it's raining or it is expected to rain in the following hours.

    \item Waste management can eliminate the cost of unnecessary waste collections and therefore reduce the carbon footprint. Data gathered can then help to identifying cost-effective itineraries to collect waste and eventually lower overall transportation and staff costs.
\end{itemize}

All this use cases refine the efficiency of the municipal workforce and help the town council to reduce costs and improve the environment sustainability in the long term.

\subsection{Industry}
\label{subsec:stateofart:areas:industry}

According to \cite{iiot}, ``the Industrial \gls{IoT} provides a way to get better visibility and insight into the company's operations and assets", therefore this leads to ``operational efficiency gains and accelerated productivity, which results in reduced unplanned downtime and optimized efficiency, and thereby profits"".
It is comprised of several use cases (\cite{iiot-cases}) such as:

\begin{itemize}
    \item Predictive Maintenance: real-time monitoring of equipment conditions and applied data analytics can help a company to significantly decrease operational expenditures. ``Other potential advantages include increased equipment lifetime, increased plant safety and fewer accidents with negative environmental impact"" (\cite{iiot-cases});
    \item Smart metering: real-time monitoring of energy, water or natural gas consumption of a building can reduce operating expenses by managing manual operations remotely, reduce energy theft and improve forecasting and streamline power-consumption (\cite{metering-sierra});
    \item Asset tracking: real-time monitoring of resources helps "to easily locate and monitor key assets, along the supply chain (e.g. raw materials, final products and containers) to optimize logistics, maintain inventory levels, prevent quality issues and detect theft" (\cite{iiot-cases}).
    \item Connected vehicles: computer-enhanced vehicles that automate many normal driving tasks can lower crash rates, and help decreasing the number of vehicles a company needs to function.
    \item Fleet management: real-time monitoring of vehicles location and conditions can help ``improving efficiency and productivity while reducing overall transportation and staff costs"" (\cite{iiot-cases}).
\end{itemize}

As we can see from the list above, the Industrial \gls{IoT} sector is focused on business efficiency and staff safety, which, as a side effect, brings environmental benefits.

\subsection{Healthcare}
\label{subsec:stateofart:areas:healthcare}

According to \cite{FIROUZI2018583} new opportunities are now arising as a result of fast-paced expansion in the areas of the \gls{IoT} and Big Data for healthcare industries. People across the globe have begun to adopt wearable biosensors, whose data is feed into the new emerging individualized health applications.
This sector incorporates numerous use cases (\cite{iot-healthcare}) such as:

\begin{itemize}
    \item Remote Healthcare Monitoring: real-time monitoring of a patient conditions such as pulse rate and heartbeat can prevent unwanted deaths;
    \item Drug management: medicine monitoring and reminder system can help the elderly to take medicine on time;
    \item Employee health management: real-time monitoring of employee's state can predict burnouts and increase a workforce productivity;
\end{itemize}

The benefits these use cases provide are a more convenient lifestyle, improvement of one life's quality, reduction in costs and increased survival rates of patients (\cite{iot-healthcare}).

\subsection{Smart Homes}
\label{subsec:stateofart:areas:home}

Visions of smart homes have long caught the attention of researchers and considerable effort has been put toward enabling home automation. However, these technologies have not been widely adopted despite being available for over three decades (\cite{iot-smarthomes}).
Based on \cite{smarthome-review} most home automation services offer the following use cases:

\begin{itemize}
    \item Smart Lighting: remote and automated control of lights inside a house can help to decrease energy wasted;
    \item Smart Air Conditioning: remote and automated control of air conditioners can keep the house comfortable while minimizing the energy wasted;
    \item Remote health monitoring: when dealing with the elderly, complex smart systems can anticipate their needs without direct human intervention;
    \item Device Automation: smart systems can turn the lights off when no one is home, open the door when an identified person arrives and must more, improving the overall comfort of the residents.
\end{itemize}

A smart home delivers various benefits such as reducing energy waste, comfort, allowing remote control of the house, monitoring of elderly patients and easy communication with health institutions (\cite{smarthome-review}).

\subsection{Open Challenges}
\label{subsec:stateofart:arch:challenges}

Even though it seems \gls{IoT} is the obvious next step for the industry, healthcare, everyone's home, public spaces/services and everything else there are some obstacles to overcome.

One of the big challenges ahead of everyone is related with antiquated ideas, tools and processes still in use today.
Each of the use cases above mentioned require a big shift in how a company works since it demands a modernization of the organization infrastructure.
\cite{tapscott2006wikinomics}, explained that ``In an age where mass collaboration can reshape an industry overnight, the old hierarchical ways of organizing work and innovation do not afford the level of agility, creativity, and connectivity that companies require to remain competitive in today's environment"".

According to \cite{7073822} this are the most important challenges regarding \gls{IoT} applications:

\begin{itemize}
    \item Technological Interoperability: achieving a seamless interaction between devices and people with devices (according to \cite{al2016iot} there's a lack of standardization in \gls{IoT} devices and technologies);
    \item Semantic Interoperability: guaranty that the devices interpret the shared information
correctly and act accordingly (improvements have to be made regarding distributed ontologies, semantic web, or semantic device discovery);
    \item Security and Privacy: improving data integrity, unique device identification, encryption and implement proper data/device ownership for legal/liability issues;
    \item Smart Things: ultra low power circuits and devices capable of tolerating harsh environments have to be developed;
    \item Resilience and Reliability: in industrial environments or in emergency use cases temporary outages cannot be accepted.
\end{itemize}

According to the author this challenges substantially lingered the growth of \gls{IoT}, an area that was expected to have a much bigger impact in day-to-day life of everyone. According to \cite{iot-cisco-prediction} there would be 50 billion of devices connected to the Internet by 2020 but \cite{statista-number-devices} reported only 8.74 billion of connected devices.

\cite{noura2019interoperability} introduced more issues in \gls{IoT} related to interoperability from different perspectives:

\begin{itemize}
    \item Device interoperability: concerned with the exchange of information between heterogeneous devices and the ability to integrate new devices into any \gls{IoT} platform;
    \item Network interoperability: concerned with information addressing, routing, security, resource optimization, \gls{QoS} and mobility support;
    \item Syntactical interoperability: concerned with the format and structure of the information exchanged between heterogeneous systems;
    \item Semantic interoperability: concerned with the meaning behind the information exchanged, heterogeneous devices can, for example, work with diverse unit measurements;
    \item Platform interoperability: concerned with heterogeneous platforms that use diverse programming languages, \gls{OS} and software architectures.
\end{itemize}

For \gls{IoT} Technologies to deliver on the promises made by companies like Cisco or Gartner, these barriers must be surpassed.

\section{Synopsis}
\label{sec:stateofart:synopsis}

This chapter presented the big theme surrounding this work: \gls{IoT}.

In the \gls{IoT} section some business cases relevant for this work were introduced. Besides these, several solutions currently in the market were presented.
The technologies usually used to tackle the challenges related to \gls{IoT} were presented in the \nameref{sec:stateofart:tech} section, these were: (i) Asynchronous Communication, (ii) Data Processing and (iii) Data Storage.

In the following chapter, \nameref{chap:requirements}, some of the business cases and challenges discussed here will be tackled.
