\chapter{Design}
\label{chap:design}

In order to describe the system in detail at the architectural level, an approach
based on the combination of two models, C4 and 4+1 will eb followed.

\section{Reference Architecture}
\label{sec:design:ref_architecture}

\section{System Scopes}
\label{sec:design:system_scopes}

The system designed can be divided is three main scopes as disclosed in the Figure~\ref{fig:design:system_scopes:scopes}.

\begin{figure}[H]
    \centering
\resizebox{\columnwidth}{!}
{
   \input{assets/diagrams/scopes.latex}
}
\caption[System Scopes]{System Scopes}
\label{fig:design:system_scopes:scopes}
\end{figure}

The \textbf{Configuration Scope} adheres to the configuration and visualization of internal processes/contexts. This processes, such as: (i) data decoders, (ii) device inventory, (iii) warning rules definition and (iv) device ownership, are related to the \textbf{Data Flow Scope}. It is also possible to manage tenants' access and permissions in the system in this scope.

The \textbf{Data Flow Scope} behaves according to what is defined in the \textbf{Configuration Scope} and acts as a pipeline where raw device data goes though various stages till it is sanitized and ready to be supplied to the \textbf{Services Scope}. The \textbf{Data Flow Scope} is where internal processes occur, such as: (i) data transformation, (ii) data enrichment, (iii) data validation, (iv) data ownership clarification and (v) warnings dispatching.

The \textbf{Services Scope} is comprised of services that present and act according to the sanitized data that was supplied to them. This services applicability range from (i) smart irrigation, (ii) fleet management, (iii) fire detention, (iv) physical security access monitoring, (v) air quality monitoring and anything else deemed interesting.

\subsection{Configuration Scope}
\label{subsec:design:system_scopes:configuration_scope}

The \textbf{Configuration Scope} is responsible for managing the following contexts:

\begin{itemize}
   \item \textbf{Data Processor}: manages simple data mappers;
   \item \textbf{Data Decoder}: manages scripts to transform data;
   \item \textbf{Device Management}: manages device information such as name, metadata, static data and other notions;
   \item \textbf{Identity Management}: manages device ownership and users permissions;
   \item \textbf{Rule Management}: manages scripts that consume device data and produce alerts;
\end{itemize}

Each context allows an authorized user to manage its resources, e.g. the data processor context manages the creation, deletion and renovation of data mappers.

This operations require various verifications, alter the system internal state and are therefore prolonged operations.

\subsection{Data Flow Scope}
\label{subsec:design:system_scopes:data_flow_scope}

The \textbf{Data Flow Scope} is responsible for processing incoming data according to what is defined in the \textbf{Configuration Scope}. Both scopes share the same contexts, apart from the data validation context (only present in this scope).

The data validation context preforms basic data filtering based on static rules, e.g. battery percentage reported has to be in between 0 and 100.

This scope applies changes to the device data that flows though the system. This changes are stateless and don't change the overall state of the internal system state.

This scope was decoupled from the \textbf{Configuration Scope} even though they both work with the same contexts. The decision was taken based on the pretext that despite the similarities in context the operation/business processes of this two scopes were conflicting.

The \textbf{Configuration Scope} requires scarce but heavy computations that alter the internal system state while the \textbf{Data Flow Scope} requires plentiful but light computations that don't alter the internal system state as summarized in the Table~\ref{tab:design:system_scopes:data_flow_scope:comparison}.

\begin{table}[!ht]
   \centering
   \begin{tabular}{|l|l|l|}
   \hline
   \textbf{Comparison of Operations} & \textbf{Configuration Scope} & \textbf{Data Flow Scope} \\ \hline
       Alter internal system state & yes & no \\ \hline
       Alter sensor data & no & yes \\ \hline
       Required computation power/time & high & low \\ \hline
       Frequency of usage & low & high \\ \hline
   \end{tabular}
   \caption[Comparison of Operations in Data Flow and Configuration Scopes]{Comparison of Operations in Data Flow and Configuration Scopes}
   \label{tab:design:system_scopes:data_flow_scope:comparison}
\end{table}

Due to this discrepancy it's expected for each scope to have different requirements regarding horizontal scaling. With the addition of more devices to the platform, and subsequently higher ingress volume, \textbf{Data Flow Scope} will need to scale. Since the \textbf{Configuration Scope} is intended mostly for the manager of the platform, never more than 10 concurrent users, the need to scale is smaller.

\subsection{Service Scope}
\label{subsec:design:system_scopes:service_scope}

\section{Domain}
\label{sec:design:domain}

\subsection{Concepts}
\label{subsec:design:domain:concepts}

\subsection{Shared Model}
\label{subsec:design:domain:shared_model}

\subsection{Bounded Contexts}
\label{subsec:design:domain:bounded_contexts}

\section{Architectural Design}
\label{sec:design:architecture}

In order to describe the system in detail at the architectural level, an approach
based on the combination of two models, C4 (\cite{c4model-site}) and 4+1  will be followed.

The 4+1 View Model (\cite{4plus1model}), proposes the description of the system through complementary views thus allowing to separately analyze the requirements of various software stakeholders, such as users, system administrators, project managers, architects, and programmers.

The five views are thus defined as follows:
\begin{itemize}
   \item \textbf{Logical view}: relative to the aspects of the software aimed at responding to business challenges;
   \item \textbf{Process view}: relative to the process flow or interactions within the system;
   \item \textbf{Development view}: relative to the organization of the software in its development environment;
   \item \textbf{Physical view}: relative to the mapping of the various components of the software in hardware, i.e. where the software is executed;
   \item \textbf{Scenario view}: related to the association of business processes with actors capable of triggering them.
\end{itemize}

The C4 Model (\cite{c4model-site}, \cite{c4model}) advocates describing software through four levels of abstraction:
(i) system, (ii) container, (iii) component, (iv) code. Each level adopts a finer granularity than the level that precedes it, thus giving access to more details of a smaller portion of the system.
These levels can be likened to maps, e.g. the system view corresponds to the globe, the container corresponds to the map of each continent, the component view corresponds to the map of each of each country, and the code view to the map of roads and neighborhoods in each city.

Different levels allow you to tell different stories to different audiences.

The levels are defined as follows:
\begin{itemize}
   \item \textbf{Level 1}: Description (context) of the system as a whole;
   \item \textbf{Level 2}: Description of system containers;
   \item \textbf{Level 3}: Description of components of the containers;
   \item \textbf{Level 4}: Description of the code or smaller parts of the components.
\end{itemize}

These two models can be said to expand along distinct axes, with the
C4 Model presenting the system with different levels of detail and the 4+1 View Model
presents the system from different perspectives. By combining the two models it becomes possible to represent the system from several perspectives, each with various levels of detail.
To visually model/represent the ideas designed and alternatives considered, the Unified Modeling Language (UML) was used.

In the following sections only combinations of perspectives and level deemed relevant for the design of the solution are presented.

The C4 level 4, code, will not be exhibited.

\subsection{C4 Level 1 - Context}
\label{subsec:design:architecture:context}

The context level aims at introducing the system as a whole. The external
systems and users that communicate/interact with the system, \textbf{Sensae}, are demonstrated.
Throughout this section the relevant C4 views of level 1 (context level) are presented.

\subsubsection*{Context Level - Logical View}
\label{subsubsec:design:architecture:context:logical}

The logical view of the system is introduced here, complete but not detailed, in order to answer the use cases and requirements discussed in \textbf{***TODO***}. This takes into account the interactions of the platform with external systems and its interaction with the various actors of the system (Figure~\ref{fig:design:architecture:context:logical:diagram}).

\begin{figure}[H]
   \centering
   \resizebox{\columnwidth}{!}
   {
      \input{assets/diagrams/logical-view-level1.latex}
   }
   \caption[Context Level - Logical View Diagram]{Context Level - Logical View Diagram}
   \label{fig:design:architecture:context:logical:diagram}
\end{figure}

The external systems and its functions are as follows:
\begin{itemize}
   \item \textbf{Helium Console}: Device data hub;
   \item \textbf{Azure Active Directory}: User authentication/identity;
   \item \textbf{Google Identity Platform}: User authentication/identity;
   \item \textbf{Twilio Platform}: SMS delivery;
   \item \textbf{Google Email Platform}: Email delivery;
\end{itemize}

The reason behind the use of external authentication/identity services is described in the Section~\ref{subsec:design:alternatives:auth}.

\subsubsection*{Context Level - Physical View}
\label{subsubsec:design:architecture:context:physical}

Next is the physical view (Figure~\ref{fig:design:architecture:context:physical:diagram}), intended to familiarize the reader with the idealized production environment.

\begin{figure}[H]
   \centering
   \resizebox{\columnwidth}{!}
   {
      \input{assets/diagrams/physical-view-level1.latex}
   }
   \caption[Context Level - Physical View Diagram]{Context Level - Physical View Diagram}
   \label{fig:design:architecture:context:physical:diagram}
\end{figure}

Due to time constrains the environment was not deployed to \textit{Kubernetes} and the solution is instead orchestrated using \textit{Docker Compose} in a single node/server.

\subsubsection*{Context Level - Development View}
\label{subsubsec:design:architecture:context:development}

Next is the development view (Figure~\ref{fig:design:architecture:context:development:diagram}), intended to familiarize the reader with how the software is organized.

\begin{figure}[H]
   \centering
      \input{assets/diagrams/development-view-level1.latex}
   \caption[Context Level - Development View Diagram]{Context Level - Development View Diagram}
   \label{fig:design:architecture:context:development:diagram}
\end{figure}

The package \textit{iot-core} contains the shared model discussed in the Section~\ref{subsec:design:domain:shared_model}, and functions to define what type of device data/internal state one wants to subscribe to and publish (discussed in Section~\ref{subsec:design:domain:concepts}).

\subsubsection*{Context Level - Synopsis}
\label{subsubsec:design:architecture:context:synopsis}

The process view was not represented since at this level the interactions between the system, actors and external systems, are too abstract to be relevant for the reader.

\subsection{C4 Level 2 - Containers}
\label{subsec:design:architecture:containers}

\subsection{C4 Level 3 - Components}
\label{subsec:design:architecture:components}

\section{Architectural Alternatives Discussed}
\label{sec:design:alternatives}

\subsection{User Authorization/Authentication}
\label{subsec:design:alternatives:auth}

\subsection{Internal Communication}
\label{subsec:design:alternatives:communication}

\section{Synopsis}
\label{sec:design:synopsis}
